\begin{description}

\item[Background image] An image that is either a geographic map of the campus or a floor-plan of a building.

\item[CLAN] ``Campus Location and Navigation''. The project name.

\item[Floor plan map] \textit{Map} of one floor of the interior of a building.

\item[GPS] ``Global Positioning System'', a global satellite system to find out the own position.

\item[GUI] ``Graphical user interface'' is a human-computer interface (i.e., a way for humans to interact with computers) that uses windows, icons and menus and which can be manipulated by a mouse (and often to a limited extent by a keyboard as well).

\item[Map] Combination of an background image and a routing graph.

\item[MVC-Architecture] Model-View-Controller. A software architecture principle that describes the separation of the presentation (View), input handling (controller) and data storage and manipulation (model).

%\item[Places, Locations] Rooms, lecture halls etc.

\item[Property]
Data attached to edges and nodes which contain additional information. This data is used to allow routing after certain criteria (e.g. without stairs) and to display information about the found route.

\item[Routing Graph] A graph used for routing. In this application each vertex represents a searchable entity, e.g entrances of a building, an office or a coffee machine, while edges represents the ways between the vertices, e.g. roads, walk paths, stairs or corridors.

\item[Search field] The \emph{From:} or the \emph{To:} text field (See \textit{GUI}, Routing view).

\item[Searchable information] The name, number or address of a building or room.

\item[XML] ``Extensible Markup Language'', a human readable file format to store hierarchical data. It describes elements (e.g. a vertex) which may contain additionally information (e.g. the position of the vertex).

\end{description}
