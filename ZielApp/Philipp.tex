\section{Philipp}

It is often difficult for KIT visitors and new students to find the buildings, rooms or the ways between them on the campus. Our product will simplify this task, leading to better time efficiency and less stress for everyone.

The general goal of this project is the design and implementation of a routing system for the KIT campus. The system will be able to compute and to display the shortest path between a starting point and a destination. It will be possible to route from and to buildings and rooms inside buildings. For the latter, the system will display a floor plan map of the building. It will also be possible to search just for the destination, without computing a route from some other point. \\

\noindent On the administration side, the system will provide a graphical use interface (\textit{GUI}) to modify the routing graph if needed (in case of construction works or if some more details are needed) or create it from scratch. It will be possible to easily add/delete buildings with or without routing information of the interior. Moreover, it will be possible to add vertices and edges with information that is necessary to do the routing.

The product will be shipped with \textit{floor plan maps} of at least two buildings of ``KIT Campus Süd'' and a \textit{map} of the campus exterior.


%Both routing and administration will share a component for map visualization.

%\subsection{Routing}
%
%The routing starts with the user input. It will be easy for the user to specify a building, a room or a lecture hall as starting point or destination. The system will compute a shortest path from the starting point to the destination. The methods of user interaction are more closely described in the mandatory criteria section of this document. \\
%
%\subsection{Map Data - Administration Tool}
%
%Administration tool will be used to create the map data. It will be possible to add and edit every piece of data that is or can be used by the routing application. The functionality of the administration tool is described
%in detail in the mandatory criteria section of this document.



\subsection{Vorschlag 1}


\textcolor{red}{TODO:} Anschaun, ob es schonmal so ein APP existiert. Rashad hat irgendwie gemeint, dass es da sowas schon existiert, z.B mycycle, wir haben es noch nicht verifiziert. 

\vspace{0.2cm}

Doch, sowas existiert schon, Rashad hat gerade was gefunden. Die App heißt \emph{Money Journal Light}. Einfach in Android-App Store nach \textbf{Budget Manager App} suchen. z.B \emph{Expense Manager}, die kostenlos zu erhalten ist. 


\subsubsection{Produktziel} 
Die wichtigsten Punkte sind die Folgenden
\begin{itemize}
\item App wird Diagramm zeigen, wo man nach bestimmten Kategorien seine Ausgaben ansehen kann, wie die sich in letzter Zeit entwickelt haben bzw. geändert haben, so wie typische Statistik.
\item Wenn man z.B Milch braucht, 
\end{itemize}


\subsection{Vorschlag 2}
Ein Pseudezufallsgenerator, tipisches Use Case. Wenn jetzt z.B wir uns Eis kaufen wollen, aber wir können uns kaum entscheiden, was, dann tippen wir die Namen von den 3 Optionen, und dann man schüttelt die App, dann bekommt man eine zufällige ANtwort, eine von 3. Genau, oder wenn wir irgendwas machen wollen und wir wissen, wer das macht, z.B ob Rashad, oder Philipp, Kanan das macht, dann gibt man es in dieser App ein und dann schüttelt man Handy, dann man kriegt, die zufällige ANtwort. 


\subsubsection{Zwei Modi}
Einmal alle ENtscheidungsmöglichkeiten eintippen so wie oben beschrieben. Und einmal gib mir eine Zahl zwischen 1 und n, z.B wenn man im Restaurant ist, und die Gerichte durchnummeriert sind, dann kriegt man von der App eine Zufallszahl und somit entscheidet sich für Gericht mit Ausgabenummer. 


\subsubsection{Use Case im Unterricht}
Wenn eine Lehrerin irgendwie einen Schüler oder Schülerin fragen will und sie es mit Zufall machen will, damit es fair ist, dann könnte sie auch unsere App verwenden. An der Stelle muss man sich mit dem Thema Zufallszahl auseinandersetzen. Denn diese Zufallszahl muss echte Zufall sein. 


\subsubsection{Ergänzung von Rashad}
Mit Datenbank könnte man es besser organisieren. 


\subsubsection{Leider existiert schon}
Wir haben grade bei App Store eine App gefudnen, wo so eine Funktionalität angeboten wird. Z.B siehe \textbf{Randomizer}. 



\subsection{Vorschlag 3}

Problem von Zeichenprogrammen für Handys/Tablets: Finger zu grob um feine Details zu zeichnen. 

Lösungsmöglichkeiten von unserer neuartigen Zeichenapp:
Linie wird nicht unter dem Finger gezeichnet (wo man nichts sieht), sondern neben dem Finger wird eine Pfeilspitze angezeigt. Und dort wo die Pfeilspitze hinzeigt wird dann gezeichnet.
Um nur zu zielen könnte man nur einen Finger benutzen zum zeichnen dann einen zweiten Finger aufsetzen und mit zwei Fingern zeichnen.

andere Lösungsmöglichkeit:
Cursor, der sich unabhängig vom der (absoluten) Position des Fingers bewegt.

