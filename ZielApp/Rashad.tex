\section{Rashad}

\subsection{Vorschlag 1 - SportStatistiken}
Ein App, mit dem man Statistik über den Spieler haben kann, beispielsweise sagen wir mal so, wenn man z.B Basketball spielt und man will seine Punkte irgendwo eintragen, um später gucken zu können, wie die seine Leistung sich in der letzten Zeit entwickelt hat. Ob er mehr Misses als Hits hatte oder andersrum. 


\subsubsection{Produktbeschreibung}
Mit dieser App kann man Statistik über sich selbst oder die Spieler machen, z.B wenn wir Fußball spielen, und die Statistiken über den verschiedenen Spieler haben wollen, dann könnte man diese App verwenden. Insbesondere wird die App fuer die Trainers sehr nutzlich.

\vspace{0.2cm}

\subsubsection{Contras}
\begin{itemize}
\item Was würde es einem Client bringen? Wäre bereit für so eine App zu zahlen?
\item Ob es sich überhaupt verallgemeinern lässt? Es gibt ja verschieden Spiele.
\item Verkauf der App ist begraenzt, weil nicht jede das kauft. Potenzielle Kunden sind Trainers die Leute, die selbst mit einer Art Sport beschaeftigen. 

\subsection{Vorschlag 2 - Scientific Document Creator}
Es ist eine WYSIWYG-App (\"What You See Is What You Get\“, \"Was du siehst, ist das, was du bekommst.\“) um Wissenschaftliche Documenten zu erschaffen. Wir nutzen TeX bash Komando um zu konvertieren ins PDF, oder HTML um ein ePUB zu erschaffen. GUI muss sehr einfach sein, so dass, jeder Wissenschaftler/in die ohne Programmier/HTML oder andere kenntnisse verwenden kann. Unterschied zu Word oder aehnliches Program, hier gibt es einfache Formel- und Grapheneingabensystem. In Zukunft auch Unterschtuetzung der Spracheingabe.

\vspace{0.2cm}

\subsubsection{Contras}
\begin{itemize}
\item Was würde es einem Client bringen? Wäre bereit für so eine App zu zahlen?
\item Ob es sich überhaupt verallgemeinern lässt? Es gibt ja verschieden Spiele.
\item Verkauf der App ist begraenzt, weil nicht jede das kauft. Potenzielle Kunden sind Wissenschafftler und die Leute, die ein Buch schreiben moechten.

\subsection{Vorschlag 3 - I am Robot}

Eine App, die spass macht. Die App wendet die Kamera des Handys an, aber zeigt alles so, wie es ein Robot aus Sci-Fi sieht. Es schliesst sich auf die Koepfe der Personen, macht kleine Foto davon,\"analisiert" (nicht tatsaechlich aber) dann zeigt Informationen ueber die Foto, wie Verbrecher, gesucht oder normaler Buerger. An der ecke der Bildschirm werden immer DOS-Typ informationen ausgedrueckt.
Auch ein virtuelles Waffensysten soll beruecksichtigt werden. Z.B. Man soll moeglichkeit haben, mit seinem hand jemanden zu zeigen und system schiest sich auf diese Person als Ziel.

\subsubsection{Contras}
\begin{itemize}
\item Was würde es einem Client bringen? Wäre bereit für so eine App zu zahlen?
\item - Es macht spass, man fuehlt sich wie ein Robot. 1 euro fuer so eine Spass ist zuverlaessig.
\item Verkauf der App ist unbegraenzt. Potenzielle Kunden sind die Leute, die ungewoehliche spass haben moechten, insbesondere Kinder.

\end{itemize}