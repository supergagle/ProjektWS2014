\section{Rashad}

\subsection{Vorschlag 1 - SportStatistiken}
Ein App, mit dem man Statistik über den Spieler haben kann, beispielsweise sagen wir mal so, wenn man z.B Basketball spielt und man will seine Punkte irgendwo eintragen, um später gucken zu können, wie die seine Leistung sich in der letzten Zeit entwickelt hat. Ob er mehr Misses als Hits hatte oder andersrum. 


\subsubsection{Produktbeschreibung}
Mit dieser App kann man Statistik über sich selbst oder die Spieler machen, z.B wenn wir Fußball spielen, und die Statistiken über den verschiedenen Spieler haben wollen, dann könnte man diese App verwenden. Insbesondere wird die App f"ur die Trainers sehr n"utzlich.

\vspace{0.2cm}

\subsubsection{Contras}
\begin{itemize}
\item Was würde es einem Client bringen? Wäre bereit für so eine App zu zahlen?
\item Ob es sich überhaupt verallgemeinern lässt? Es gibt ja verschieden Spiele.
\item Verkauf der App ist begraenzt, weil nicht jede das kauft. Potenzielle Kunden sind Trainers die Leute, die selbst mit einer Art Sport besch"aftigen. 
\end{itemize}

\subsection{Vorschlag 2 - Scientific Document Creator}
Es ist eine WYSIWYG-App (``What You See Is What You Get``, ``Was du siehst, ist das, was du bekommst.``) um Wissenschaftliche Documenten zu erschaffen. Wir nutzen TeX bash Komando um zu konvertieren ins PDF, oder HTML um ein ePUB zu erschaffen. GUI muss sehr einfach sein, so dass, jeder Wissenschaftler/in die ohne Programmier/HTML oder andere kenntnisse verwenden kann. Unterschied zu Word oder aehnliches Program, hier gibt es einfache Formel- und Grapheneingabensystem. In Zukunft soll auch Unterscht"utzung der Spracheingabe m"oglich sein.

\vspace{0.2cm}

\subsubsection{Contras}
\begin{itemize}
\item Was würde es einem Client bringen? Wäre bereit für so eine App zu zahlen?
\item Ob es sich überhaupt verallgemeinern lässt?
\item Verkauf der App ist begraenzt, weil nicht jede das kauft. Potenzielle Kunden sind Wissenschafftler und die Leute, die ein Buch schreiben m"ochten.
\end{itemize}

\subsection{Vorschlag 3 - I am Robot}

Eine App, die spass macht. Die App wendet die Kamera des Handys an, aber zeigt alles so, wie es ein Robot aus Sci-Fi sieht. Es schliesst sich auf die K"opfe der Personen, macht kleine Foto davon, ``analisiert`` (nicht tats"achlich aber) dann zeigt Informationen "uber die Foto, wie das ist ein Verbrecher, gesucht oder normaler B"urger. An der ecke der Bildschirm werden immer DOS-Typ informationen ausgedr"uckt.
Auch ein virtuelles Waffensysten soll ber"ucksichtigt werden. Z.B. Man soll m"oglichkeit haben, mit seinem hand jemanden zu zeigen und system schliest sich auf diese Person als Ziel.

\subsubsection{Contras}
\begin{itemize}
\item Was würde es einem Client bringen? Wäre bereit für so eine App zu zahlen?
\item - Es macht spass, man f"uhlt sich wie ein Robot. 1 euro f"ur so eine Spass ist zuverl"assig.
\item Verkauf der App ist unbegr"anzt. Potenzielle Kunden sind die Leute, die ungewoehliche spass haben m"ochten, insbesondere Kinder.
\end{itemize}

\subsection{Vorschlag 4 - VirtualReality Typ Spiel}

Wieder eine App, die spass macht. Die App wendet die Kamera des Handys an, aber zeigt alles so, als du in einer anderen Welt bist. Z.B. Coins auf der Strasse, die man als Mario sammeln kann, oder man kann auf der Strasse wie beim ``Urban Surfers`` surfen kann. Oder Ein MMORPG, abaer statt der 3D-Welt, nutzen wir reale Welt durch Kamera.

\subsubsection{In Zukunft} In Zukunft soll unterst"utzung virtueller Klaider m"oglich sein. Wenn man jemanden durch Kamera sieht und wenn diese/r jemand auch ein/e Spieler/in ist, dann App zeigt ihn/sie in virtuellen Kleidern, nicht in Echten. Auch Quests in realer Welt sollen unterst"utzt werden.

\subsubsection{Contras}
\begin{itemize}
\item Was würde es einem Client bringen? Wäre bereit für so eine App zu zahlen?
\item - Es macht spass, man f"uhlt sich wie ein Held/in. 1 euro f"ur so eine Spass ist zuverl"assig.
\item Verkauf der App ist unbegr"anzt. Potenzielle Kunden sind die Leute, die ungewoehliche spass haben m"ochten.
\end{itemize}