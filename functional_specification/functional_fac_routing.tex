%\emph{Necessary for the facultative criteria.}

\subsubsection{Routing}
\begin{numerate}[FFR]
%\item \textbf{/FR080/]
%The user should be able to get the application as an Android App or as a Java Applet???
\item[Show Current Position]
It is possible to show the current position of the user on the \textit{map} by using a \textit{GPS} signal.
\item[Search by GPS Coordinates]
To search for a location, the user can also type its \textit{GPS} coordinates into the text field.
\item[Navigation Using GPS]
The user can navigate to a destination from his current position by using \textit{GPS} to get his position.
\item[Switch Routing Mode]
The user can switch the routing mode between pedestrian, bicycle and wheelchair.
%\item[Textbased Route Output]
%Textbased output of the route can be viewed.   \textcolor{red}{WTF? see FMR170}
\item[Show Estimated Travel Time]
The user can enter his own average speed in a provided dialog. After he has done this, the routing tool will calculate and display
the estimated time it will take the user to traverse the routes. If no average speed is given no time will be displayed.
\item[Bookmark Routes]
The user can mark a route as a favorite, so the corresponding search query will be saved and can be reused later.
%\item \textbf{/FR15[]
%There will be the possibility to calculate and show different alternative routes.
\item[Select Waypoints by Clicking]
The user can mark different points on the \textit{map} and then request a route from a start point to a destination over these waypoints.
\item[Search Point of Interest]
There is list of categories from points of interests (e.g., restrooms). The user can select one categorie and then he will see the nearest to him marked on the map.
\item[Filters]
The tool offers more efficient routing options using filters (e.g., use no stairs in the route).
\end{numerate}