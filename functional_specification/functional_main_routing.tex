%\emph{Necessary for the mandatory criteria.}

\subsubsection{Routing Tool}
\paragraph{Displaying the map}
\begin{numerate}[FMR]
	\item[Show Campus Map]
	A \textit{map} of the campus is displayed.
	\item[Show Floor plan map]
	A \textit{floor plan map} of a building can be displayed.
	%\item[Highlight Subgraph]
	%The given subgraph is displayed on the map. This can in particular be used to display a location or a route.
	\item[Show Route]
	A calculated route is displayed on the \textit{background image} in a red color.
	\item[Show Way under Building]
	A way passing under a building is displayed in a blue color.
	\item[Select Point]
	The user is able to select a point on the \textit{map}, which will be marked. This can be used to select start and destination by clicking on the \textit{map}.


\end{numerate}

\paragraph{Navigation in the map}
\begin{numerate}[FMR]
	\item[Zoom]
	The user can upscale and downscale the displayed section of the \textit{background image} with the corresponding buttons.
	%The user can view a section of the map and is able to upscale and downscale this section.
	\item[Pan]
	The user can move the displayed section of the \textit{background image} by using the mouse.
	%The user can view a section of the map and is able to move this section.
	\item[Enter Building]
	The user can enter a building by performing a mouse double-clicking on it.
	\item[Switch Floor plan map inside a Building]
	%If the user has entered a building, he can move a slider to view the different floorplans of the building and switch between them.
	If the user has entered a building, he can switch his view between different \textit{floor plan maps} of the building by moving a special slider provided.
	\item[Leave Building]
	The user can leave a building by clicking the provided ``outside map'' icon in the corner of the \textit{floor plan map} or by double-clicking outside of the \textit{floor plan map}.
\end{numerate}

\paragraph{Searching and Routing}

\begin{numerate}[FMR]
	\item[Search Location]
The user can search for a location without requesting a route to it by typing the data of it either in the start or in the destination \textit{search field} (/FMR140/).

	\item[Find Search Suggestions]
	When the user types into a \textit{search field}, the current input is compared to the \textit{searchable information} of each building, to find possible completions. The resulting list is displayed.

	%\item[Find Nearest Vertex]Find nearest vertex in the underlying graph of the map by clicking on the map

\item[Compute Route]
		The user can type in a starting point and an end point into two different text fields (/FMR140/). By clicking a button,
		the computation of the shortest route (for pedestrians) between these two points starts. When computation is completed the computed route is displayed. If the destination is a building, the route will end at the nearest entrance of that building.


\item[Search Field Input]
%The data in the textfields can be: a building number, a building number combined with a room number, an address of a building (road and housenumber) or the name of the building/room/location
%(Does this not belong to product data?????)
The user can search a location by name, building number, room number and address.

%\item[Display Route]
%Display route (uses Highlight subgraph -> need to be formulated as separate functionality ???)
%\begin{addmargin}[7mm]{0mm}
			%After a route is calculated, the user can see the calculated route as a colored path on the \textit{map}. The length of that route is also displayed.
%		\end{addmargin}

%\item[Route to Nearest Entrance]


\item[Output Textbased Description]
The route can be shown in a textual form like: ``Walk for 100m. Turn right. Walk for 25m. You have reached your
destination.'' This can be done by clicking on the corresponding option.

\item[Discard Route]
			If the user wants to discard the calculated route from the \textit{map}, he can either calculate a new one that will then be shown instead the previous or
			simply use the ``Remove route'' function that the tool provides.


\end{numerate}
