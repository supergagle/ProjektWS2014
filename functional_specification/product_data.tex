\subsection{Essential Data}

\begin{numerate}[D]
	%\item The graph data in the OSM XML format. We chose this format because it's human-readable, easy to understand and easy to modify. The basic structure is as follows: \\
	%Consider the following excerpt from a map graph: \\

%%%%%%%%%%%%%%%%%%%%%%%%%%%%%%%%%%%%%%%%%%%%%%%%%%%%%
	\item  [The graph data in XML using our file structure] We chose \textit{XML} because it is human-readable, easy to 	understand and easy to modify.
			We loosely based our format on OSM XML (see \url{http://wiki.openstreetmap.org/wiki/OSM_XML}), but made some major changes to make it suit our needs.
			The basic structure of our format can be described as follows: \\
			First is the underlying \textit{background image} with a scale. The scale defines how many pixel on the screen are hundred meters
			Then is a list of all nodes. Each node can have multiple \textit{properties}. Those \textit{properties} can include
			building IDs and entrance IDs that link the entrance-nodes of buildings to their counterparts in the interior.
			Second is a list of all edges between the nodes, representing the ways on our \textit{map}.
			The edges can also have a set of \textit{properties} attached to them.
			Last is a list of buildings, which are defined as polygons and include a link to the \textit{XML}-File
			containing the \textit{map} of the building interior.
			The following example shows the representation of a \textit{routing graph} consisting of four nodes and one building.
			Two of those nodes are entrances to the building.\\

	\begin{minipage}{\textwidth}
	\lstset{language=XML}
	\begin{lstlisting}
<?xml version="1.0"?>
<campusmap version="1.0" image="map/kitmap.png" name="KIT Campus Sued" scale="163.37">
	<node id="4" x="1253" y="332">
		<tag key="building_id" value="3219" />
		<tag key="entrance_id" value="0" />
	</node>
	<node id="2" x="998" y="711">
		<tag key="building_id" value="3219" />
		<tag key="entrance_id" value="1" />
	</node>
	<node id="3" x="1200" y="345">
		<tag key="bicycle_stand" value="true"/>
	</node>
	<node id="1" x="963" y="734" />
	<edge id="7928" node_1="1" node_2="2" />
	<edge id="7122" node_1="2" node_2="3">
		<tag key="length" value="15" />
	</edge>
	<edge id="3228" node_1="4" node_2="3" />
	<building file="buildings/infobau.xml" id="3219">
		<tag key="display_name" value="Infobau, Geb. 50.34" />
		<tag key="name" value="50.34" />
		<tag key="name" value="Infobau" />
		<outline coords="93,26;234,23;323,74;23,9;8,90" id="0" />
	</building>
</campusmap>
	\end{lstlisting}
	\end{minipage}

	The following are the contents of ``infobau.xml'', representing the interior of the building. The building \textit{XML} file is split into different \textit{floor plan maps} with
	special stair-edges linking them together.

		\begin{minipage}{\textwidth}
	\lstset{language=XML}
	\begin{lstlisting}
<?xml version="1.0"?>
<building version="1.0">
	<defaultfloor floor="0" />
	<floorplan floor="0" image="buildings/infobau_f0.png">
		<node id="0" x="90" y="11">
			<tag key="entrance_id" value="0" />
		</node>
		<node id="1" x="83" y="18" />
		<node id="2" x="71" y="2" />
		<edge id="1" node_1="0" node2="1" />
		<edge id="2" node_1="1" node2="2" />
	</floorplan>
	<floorplan floor="-1" image="buildings/infobau_f-1.png">
		<node id="3" x="21" y="13">
			<tag key="entrance_id" value="1" />
		</node>
	</floorplan>
	<stairs>
		<edge id="0" node_1="2" node_2="3">
			<tag key="elevator" value="true" />
			<tag key="length" value="6" />
		</edge>
	</stairs>
</building>
	\end{lstlisting}
	\end{minipage}

	As shown in the example, key/value pairs can be assigned to nodes or ways as node-properties and way-properties, e.g. node 3 of the exterior \textit{map} has been defined as a bicycle stand.
	The details of our format will be presented in the next document.
%%%%%%%%%%%%%%%%%%%%%%%%%%%%%%%%%%%%%%%%%%%%%%%%%%%%%%

	\item[The background image of the map] Supported file formats are: PNG, JPG, GIF, BMP and PDF.

	\item[The session data] This includes information about which data has to be automatically loaded when the program starts. For example, when the user loads a \textit{map} for routing and ends the program, that \textit{map} will be automatically loaded the next time they start the program.
	\item[The help file] A help document for the program is provided, describing the functions of the routing and administration tool.
\end{numerate}

\subsection{Facultative Data}
\begin{numerate}[FD]
	\item[Filters added by the user] They can be used to restrict the \textit{routing graph} to a subgraph. For more information, see ``Facultative Criteria -- Routing Tool''.

	\item[Queries] This allows the user to bookmark favourite routes and places.

	\item[Properties] The user can add \textit{properties} that can subsequently be used for filtering.
\end{numerate}
