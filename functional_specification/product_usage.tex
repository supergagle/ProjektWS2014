This section describes how and by whom the product will be used and what they need to use it.
As they share a lot of their functionalities, the routing tool and the administration tool will be implemented within the same application. This allows
the user of the administration tool to switch to the routing tool at any time to test the map he just created.

\subsection{Routing}
\paragraph{Scopes}
Nowadays, it is not unusual for universities to be situated within extensive campuses. While this is of course convenient, it can also be
quite confusing for students, especially for freshmen. They have to look up lecture halls on big, confusing plans that are often not even up to date.
As a result, a lot of time is wasted. We seek to change this with our campus routing tool. The tool provides a well-arranged, simple yet powerful
map application that can be intuitively used by students or people working on the campus to easily find any building or calculate the shortest
route between any two points. The location or route is then visualized in a way that enables users to fully understand where they are or where they are
supposed to go with just one quick look.

\paragraph{Target Audiences}
As mentioned above, the main target audience for the routing application are university students, with the focus being on the students who are new to
the campus. No previous knowledge is needed to use it, so everyone who knows how to use a computer and has access to one will also be able to navigate
with the routing tool. Another target audience is the people working on the campus such as professors, security and cleaning personnel. They might as
well have problems finding their way on the campus when they start to work there.

\subsection{Administration Tool}
\paragraph{Scopes}
The administration tool is used to create the \textit{map} that the routing tool uses. It allows to create it from scratch or modify all the data that is
shown by the routing tool. This includes loading a \textit{background image}, creating the corresponding \textit{routing graph} and adding information to its vertices and edges.
The main goal of the administration tool is to make this difficult, tedious task simple and quick by providing an intuitive interface and automating
many of the time-consuming steps. For example, when adding the building number to one vertex of a building outline, all other vertices will automatically
receive this \textit{property} as well.

\paragraph{Target Audiences}
The \textit{map} will most likely be provided by the universities for their students. Thus, the target audience for the administration tool are people
working for the university. While the tool is mainly self-explanatory, it is also very versatile. The \textit{routing graph} can be edited in a lot of ways, so it
takes some time to completely grasp the whole functionality of the administration tool. However, this learning curve is not very steep so even the
university students will not have a problem learning how to use the tool in a short amount of time. This makes them the second target audience:
students at universities where no good, easily accessible \textit{map} exists can use the administration tool of our application to create one and share it
with their colleagues.
