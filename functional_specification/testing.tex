\subsection{Routing - Test cases}

\paragraph{Displaying the map}
\begin{numerate}[RTC]
\item Show Campus \textit{map} (/FMR010/). \\*
	\textbf{Precondition:} The user starts the program. \\*
	\textbf{Expected outcome:} The \textit{map} of the campus is displayed.
\item Select Point (/FMR050/). \\*
	\textbf{Precondition:} The user views the campus map and he selects any point on the map. \\*
	\textbf{Expected outcome:} The point selected on the \textit{map} is marked.
\item Show \textit{Floor plan map} (/FMR020/). \\*
	 \textbf{Precondition:} The user views the campus map and then selects a building on it \\*
	 \textbf{Expected outcome:} The ground \textit{floor plan map} of the selected building will be displayed
\item Show Route (/FMR030/). \\*
	\textbf{Precondition:} The user has typed a starting point and destination in the corresponding text fields and clicks the ``Go'' button (see \textit{GUI}, Routing view). \\*
	\textbf{Expected outcome:} A calculated route is displayed on the \textit{background image} in a red color.
\item Show way under building (/FMR040/). \\*
	\textbf{Precondition:} The user searched for the shortest path between ``Audimax'' and ``H.S.a.F''. \\*
	\textbf{Expected outcome:} The shortest path between selected buildings will be displayed (see /RTC040/), especially the way passing under ``Infobau'' is displayed in a blue color.
\end{numerate}

\paragraph{Navigation in the map}

\begin{numerate}[RTC]
\item Upscale and downscale the displayed section of the \textit{background image} with corresponding buttons (/FMR060/).
\item Pan the displayed section of the background image by using the mouse (/FMR070/)
\item Enter Building (/FMR080/). \\*
	\textbf{Preconditions:} The user views the exterior map and then he double-clicks on a building. \\*
	\textbf{Expected outcome:} The ground \textit{floor plan map} of the selected building replaces the exterior view.
\item Switch \textit{Floor plan map} inside a building (/FMR090/). \\*
	\textbf{Preconditions:} The user has entered the building and he moves the slider in order to switch to another floor. \\*
	\textbf{Expected outcome:} The current \textit{floor plan map} view is replaced by selected floor plan's map view.
\item Leave Building (/FMR100/). \\*
	\textbf{Precondition:} The user has entered a building and he clicks on ``outside map'' icon in the corner of the \textit{floor plan map}. \\*
	\textbf{Expected outcome:} The current \textit{floor plan map} view is replaced by map view of the exterior.
\end{numerate}


\paragraph{Searching and Routing}

\begin{numerate}[RTC]
\item Search Location (/FMR110/). \\*
	\textbf{Preconditions:} The user has typed the name of a location in the \textit{text field} and clicks the ``Go'' button (see \textit{GUI}, Routing view). \\*
	\textbf{Expected outcome:} The found location is highlighted on the \textit{map}.
\item Find Search Suggestions (/FMR120/). \\*
	\textbf{Preconditions:} The user types into a \textit{search field} the initial letters of a building's \textit{searchable information}.\\*
	\textbf{Expected outcome:} The resulting list of possible completions is displayed.
\item Compute Route (/FMR130/). \\*
	\textbf{Preconditions:} The user has typed a starting point and destination in the corresponding text fields and clicks the ``Go'' button (see \textit{GUI}, Routing view). \\*
	\textbf{Expected outcome:} The shortest path between these two points is computed and and displayed, in case the destination is a building, the route ends at the nearest entrance of that building.
\item Text Field Input (/FMR140/). \\*
	\textbf{Preconditions:} The user has typed \textit{searchable information} into a \textit{search field} and clicks the ``Go'' button (see \textit{GUI}, Routing view). \\*
	\textbf{Expected outcome:} The typed information is searched and displayed.
\item Output text based Description (/FMR150/). \\*
	\textbf{Preconditions:} The user has typed a starting point and destination in the corresponding text fields and clicks the ``Go'' button (see \textit{GUI}, Routing view). After the shortest path has been displayed, he selects ``Show text output'' form the menu. \\*
	\textbf{Expected outcome:} The route is shown in a textual form. See example in /FMR150/.
\item Discard Route (/FMR160/)\\*
	\textbf{Preconditions:} The calculated shortest route between entered points is displayed on the \textit{map} and user selects ``Remove route''. \\*
	\textbf{Expected outcome:} The displayed route is discarded and the \textit{map} of the campus is displayed.
\end{numerate}

\subsection{Administration tool - Test cases}

\begin{numerate}[ATC]

\item Display \textit{Routing Graph} (/FMA010/). \\*
	\textbf{Preconditions:} The administrator loads a map. \\*
	\textbf{Expected outcome:} The \textit{routing graph} is displayed on top of the \textit{background image}.
\item Create a \textit{map} (/FMA020/)\\*
	\textbf{Preconditions:} The administrator selects ``Create Map'' from the menu in order to create a new \textit{map} and selects a \textit{background image} to be loaded. \\*
	\textbf{Expected outcome:} \textit{The selected background image} is loaded and at the beginning there is an empty \textit{routing graph} on it.
\item Open a \textit{map} (/FMA030/). \\*
	\textbf{Preconditions:} The administrator selects ``Open'' from the menu and selects the \textit{map}. \\*
	\textbf{Expected output:} The selected \textit{map} is displayed.
\item Save a \textit{map} (/FMA040/). \\*
	\textbf{Preconditions:} The administrator views a \textit{map} and selects ``Save'' from the menu or presses CTRL + S. \\*
	\textbf{Expected outcome:} The current \textit{map} is saved.
\item Exchange \textit{Background Image} (/FMA050/). \\*
	\textbf{Preconditions:} The administrator views a \textit{map} and selects ``Load new background image'' from the menu and selects his desired \textit{background image}. \\*
	\textbf{Expected outcome:} The selected \textit{background image} is loaded and the \textit{background image} of the currently viewed \textit{map} is replaced with the new one.
\item Set Scale (/FM060/). \\*
	\textbf{Preconditions:} The administrator marks two points on the \textit{background image} and after right-clicking he selects ``Set Scale'' from menu-list and enters the distance of the marked points in meter. \\*
	\textbf{Expected outcome:} The entered information is saved and set as the scale of the current viewed \textit{map}.
\item Create Vertex (/FMA070/). \\*
	\textbf{Preconditions:} The administrator right-clicks on the \textit{background image} and selects ``Create new Vertex'' from arising menu-list. \\*
	\textbf{Expected outcome:} The new vertex is created and is available at the selected point on the current \textit{map}.
\item Create Edge (/FMR080/). \\*
	\textbf{Preconditions:} The administrator selects two vertices and then after right-clicking selects ``Add new Edge'' from arising menu-list. \\*
	\textbf{Expected outcome:} The new edge is created.
\item Create Building (/FMA090/). \\*
	\textbf{Preconditions:} The administrator draws a polygon on a map and then after right-clicking on it selects ``Save as Building'' from arising menu-list.\\*
	\textbf{Expected outcome:} The new building is created and drawn polygon represents the outline of this building.
\item Add Edges between \textit{Floor plan maps} (/FMA100/). \\*
	\textbf{Preconditions:}	The administrator has created a building with two \textit{floor plan maps}. He selects the ``Add stairs'' tool and clicks on one vertex on each floor. \\*
	\textbf{Expected outcome:} An edge between the selected floor plans is created.
\item Select Vertex or Edge (/FMA110/). \\*
	\textbf{Preconditions:} The administrator selects any vertex or any edge (test both cases) by left-clicking on it.\\*
	\textbf{Expected outcome:} The selected vertex or edge is highlighted in another color.
\item Select Building (/FMA120/). \\*
	\textbf{Preconditions:} The administrator selects any building by left-clicking on it.\\*
	\textbf{Expected outcome:} The outline of the selected building is highlighted in another color.
\item Modify \textit{Property} values (/FMA130/). \\*
	\textbf{Preconditions:} The administrator selects a vertex, edge or building (test all cases) and adds new properties or edits existing properties in the arising \textit{GUI} Window (see \emph{Figure 6}). \\*
	\textbf{Expected outcome:} The added properties or edited properties for the selected vertex, edge or building are updated.
\item Move Vertex (/FMA140/). \\*
	\textbf{Preconditions:} The administrator selects a vertex and drags it. \\*
	\textbf{Expected outcome:} The selected vertex is moved to the position the administrator drags it to.
\item Move Edge (/FMA150/). \\*
	\textbf{Preconditions:} The administrator moves end-vertices of an edge (see /ATC140/).\\*
	\textbf{Expected outcome:} The edge is moved in accordance with the movement of the end-vertices of this edge.
\item Move Building (/FMA160/). \\*
	\textbf{Preconditions:} The administrator selects a building and drags it to the desired position. \\*
	\textbf{Expected outcome:} The selected building is moved to the position the administrator drags it to.
\item Remove Vertex or Edge (/FMA170/). \\*
	\textbf{Preconditions:} The administrator selects a vertex or edge (test both cases) and presses the delete key on the keyboard.\\*
	\textbf{Expected outcome:} The selected vertex or edge is removed and is not on the currently viewed \textit{map} anymore.
\item Remove Building (/FMA180/). \\*
	\textbf{Preconditions:} The administrator selects a building and presses the delete key on the keyboard. \\*
	\textbf{Expected outcome:} The selected building is remove and is not on the currently viewed \textit{map} anymore.
\end{numerate}

\subsection{Routing - Test scenarios}

\paragraph{Test scenario 1, covering all product functions of the routing tool} ~\\

The freshman KIT student Jonas Schneider has to attend a lecture at 11:30 AM. Unfortunately, it is now 11:15  and he has no idea where the lecture hall is. He remembers he downloaded the KIT Campus Routing System \programName~the other day.\\
He boots his laptop and starts the routing program, which presents him with a \textit{map} of the KIT campus (/FMR010/). He remembers the lecture hall was called something like ``H\"orsaal am Fa...'', so he types that into the text field that says ``To:'' (/FMR140/). He is immediately presented with a list of
search suggestions, the first one being ``H\"orsaal am Fasanengarten'' (/FMR120/).
He clicks that suggestion, resulting in it getting put into the ``To:''-field. With the ``From:''-field still being empty, he clicks the button captioned ``Go'' to start the search (/FMR110/). Now the \textit{map} view shows the highlighted outline of the ``H\"orsaal am Fasanengarten'' on the \textit{map}.\\
He now decides to use the program to calculate the shortest route to the lecture hall.
He knows he is standing right in front of the ``Audimax'' lecture hall, so he wants to set this as the starting point. To do that, he looks for the ``Audimax'' on the presented campus \textit{map}, right-clicks it and selects ``From here'' (/FMR050/), putting ``Audimax'' into the ``From:''-field.
Now, with ``Audimax'' in the ``From:''-field and ``H\"orsaal am Fasanengarten'' in the ``To:''-field, he clicks the ``Go''-button again to start the calculation of the route (/FMR130/). Almost immediately, a route is displayed on the campus \textit{map} (/FMR030/ and /FMR160/).
He zooms (/FMR060/) and pans (/FMR070/) the \textit{map} a bit to study the details of the route.\\
By doing that, he notices that while most of the route is colored red, a small portion is blue, indicating that this part is going under or through a building (/FMR040/). He selects ``Route in text form'' from the program menu. For that part of the route, it says ``Pass under the building'' (/FMR150/).\\
After that, he wants to know how to get into the lecture hall. For this, he double-clicks the lecture hall to make the program enter the building (/FMR080/). The floor plan of ``H\"orsaal am Fasanengarten'' is shown (/FMR020/), along with the route, now starting at the entrance of the building's lobby and ending at the entrance of
the lecture hall which makes up almost all of the building. Earlier, he got a text from his colleagues, informing him that they are on the second floor in the lecture hall. He uses the slider next to the \textit{map} to change the floor plan view to show the second floor (/FMR090/) and notices that he can easily get up there by
using the stairs located in the lobby. \\
He clicks the iconized outside-\textit{map} to exit the building (/FMR100/) and checks the route again. He then exits the program, shuts down his laptop and starts heading to the lecture hall, still barely making it on time.

\newpage
\subsection{Administration tool - Test scenarios}

\paragraph{Test scenario 2, \textit{map} creation and editing} ~\\
The third-year KIT student Andrew Ryan noticed that many freshmen have trouble finding their way on the campus because the university administration did not provide a good map. He decides that he is going to make one, using the administration tool of the KIT Campus Routing System \programName.\\
First, he starts the program and chooses ``Create new map'' from the program menu, after which he is prompted to choose an image file to serve as the \textit{background image} for the \textit{map} (/FMA020/). He chooses the pdf-map of the campus that he found on the university website.
The main view now shows the pdf-map as \textit{background image}. He chooses ``Set scale'' from the program menu and clicks on two points on the \textit{background image} that he knows are 100 meters apart (/FMA060/). With the scale set, he selects the ``Create vertex'' tool and places ten vertices on the
\textit{background image} to represent waypoints of the \textit{graph} (/FMA070/). He chooses the ``Create edge''-tool and clicks two vertices to connect them with an edge and repeats that until he has mapped out all the ways between the ten vertices he just placed (/FMA080/).\\
Since one of the edges he placed contains a small set of stairs, he wants to somehow add that information to the \textit{graph}. To do this, he uses the ``Select''-tool and clicks the vertex, resulting in the vertex being highlighted in a different color (/FMA110/). Immediately, the properties of the vertex are shown.
He selects the ``Contains stairs'' \textit{property} and sets it to ``TRUE'' (/FMA130/).\\
Then he notices that he misplaced one vertex and connected two vertices with an edge where there shouldn't be one. He selects the ``Move''-tool and drags the vertex (and also the edges connected to it) to the correct position (/FMA140/ and /FMA150/).
To remove the misplaced edge, he selects the ``Delete edge'' tool and clicks it, resulting in the edge vanishing from the \textit{graph} (/FMA170/). \\
He is done for now, so he wants to save the \textit{map}. To do this, he chooses ``Save'' from the program menu, types ``kitmap.xml'' as filename and clicks the corresponding button to save the \textit{map} (/FMA040/). He then exits the program by closing the window.

\paragraph{Test scenario 3, building creation and editing} ~\\
Andrew wants to continue working on the \textit{map} and wants to add the building ``H\"orsaal am Fasanengarten''. He starts the admin tool of the KIT Campus Routing System \programName, chooses ``Load'' from the menu and selects ``kitmap.xml'', the file he saved earlier (see Test scenario 2) (/FMA030/).\\
The \textit{background image} and routing \textit{graph} of the \textit{map} he created are shown (/FMA010/). He selects the ``Create building'' tool and is prompted for a building name and a file name (for the XML file where the building data will be stored). He types ``H\"orsaal am Fasanengarten'' and ``hsaf\_map.xml''.
Then he clicks on the \textit{background image} to place vertices that define the outline of the building as a polygon (/FMA090/). He clicks the first building-vertex again, closing the polygon.\\
Upon closer inspection, however, he notices that he completely mis-placed most of the vertices and decides to try again. He selects the building by clicking it (/FMA120/) and presses the delete key: the building is gone (/FMA180/).\\
He then repeats the steps to create a building, this time paying more attention. With that, the building is created.\\
He now wants to add an entrance to the building. To do that, he selects ``Add entrance'' from the program menu and clicks the outline, creating a vertex with the
``entrance'' \textit{property}.\\
Then he notices that the building is not exactly at the right position, which he corrects by dragging it (/FMA160/).\\
When he enters the building by double clicking it, there is already a vertex present, representing the counterpart to the entrance outside he just created. There is no \textit{background image} yet, so only that single vertex is shown.\\
He selects ``Change background image'' from the program menu and enters ``HSaF\_FLOOR\_0.png''. The file is loaded and now shown as the new \textit{background image} behind the vertex (/FMA050/).\\
Before mapping out the ground floor, he wants to add the 1st floor. He selects ``Add floor plan map'' and when prompred for the floor level and \textit{background image} he selects ``1'' and ``HSaF\_FLOOR\_1.png''. Using the slider next to the \textit{map}, he can now switch between the floors.\\
After mapping out most of both floors, he wants to add stairs to connect them. For that, he chooses the ``Add stairs''-tool and clicks two vertices, one on floor 0 and one on floor 1 (/FMA100/). A small arrow pointing up and down respectively is now shown next to the stair-vertices.\\
Done with his work on the \textit{map} for now, he choses ``Save'' from the program menu to save his progress and exits the program by closing the window.




% -------- OLD TEST SCENARIOS
%\subsection{Routing - Test scenarios}

%\paragraph{Test scenario 1 - Request for the shortest path} ~\\
%
%\noindent \textit{The user wants to search for the shortest path between two points.}
%
%\begin{numerate}[TS1]
%\item Start the application.
%\item Zoom the \textit{map}.
%\item Select a point on the \textit{map} by clicking on the \textit{map}.
%\item Use this point as a starting point.
%\item Select destination by typing its name.
%\item Start search for the shortest path between given points.
%\item After the shortest path is displayed apply a filter and repeat the request (e.g. use no stairs).
%\item Exit the application.
%\end{numerate}
%
%%\paragraph{Test scenario 2 - Search and Display location} ~\\
%
%\noindent \textit{The user wants to search and view a building on the map.}
%
%\begin{numerate}[TS2]
%\item Start the application.
%\item Search for a building by providing initial letters of its name and using the search suggestions.
%\item Enter the building by double clicking it.
%\item View different \textit{floor plan maps} of the building.
%%\item Search a restroom in the building.
%\item Leave the building.
%\item Exit the application.
%\end{numerate}
%
%\subsection{Administration tool - Test scenarios}
%
%\begin{numerate}[TS3]
%\item Start the application.
%\item Select the ``Select map image'' option from the menu.
%\item Select a valid picture file in the ``Choose file'' dialog.
%\item Select the ``Add vertices'' tool.
%\item Click at different places on the displayed \textit{background image} to place vertices.
%\item Click on the ``Modify'' tool and use it to select a vertex.
%\item Select the empty ``name'' \textit{property} of the vertex and modify its value.
%\item Save the \textit{map}.
%\item Exit the application.
%\end{numerate}

